% % % % % % %
\subsection{A More General Level 3 Shift-and-Stack Case Study, by Mario}\label{ssec:SPCS_SAS}

A Level 3 processing case study for shift-and-stack on a large number images. By Mario.

\#1. The scheduler is configured to repeatedly (e.g., 10 times) observe a field during the same night with longer exposure than usual (e.g., 120 sec). [ and we should take the actual numbers from the TNO-DDF whitepaper; don't have the internet right now or I would].

\#2. The images are processed as regular "Level 1" products within 60 seconds, and transmitted as alerts, with results stored into the regular L1 database. This will happen automatically for all images (perhaps within some range of exposure times?).

\#3. The raw images (and all necessary calibrations), calexps, and standard L1 diffims are made available within 10 minutes to the batch system for processing with special programs-specific codes. This is the same batch system we make available to the users for running Level 3 codes [Q for us: is it? or is is the same one that's used to process calibrations? have these systems been sized?].

\#3 a). The code running the shift-and-stack processing will be externally developed and delivered, but will be installed and operated (and change controlled!) by the LSST Operations team. That is, we don't expect someone external to the ops team to babysit the code on a nightly basis. In fact, it's the opposite: once the codes are delivered, any changes will go through LSST's software change control process.

\#4. There will be a facility to trigger program-specific processing on the batch system upon the arrival of a new image (above); this processing will then be queued up for execution. We assume that the policy for processing of special programs data may give it preferential treatment relative to general-purpose L3.

\#5. Once the processing finishes, the results of will be stored to a program-specific database. No alerts (in VOEvent sense) will be issued. We will provide a generic notification facility (perhaps something as simple as an RSS feed) that new data has been made available in a certain database/data store. [This is an example where I'd want to make sure somebody within DM is planning to provide such a facility.].

\#6. The outputs stored can be special-program specific (i.e., tables with nearly arbitrary schemas -- some columns -- like ra/dec for spatial joins -- should be present in main tables). The outputs can also contain images (stored in also special-program specific repository), or custom products (treated like opaque files). The visualizations available for these (catalogs, images, arbitrary files) through the Portal will be limited (e.g., generic table visualizations or x-y plots).

\#7. When the images are made available to the batch system (step \#3), they also become available to *everyone*. I.e., someone else could also run a custom L3 pipeline on these data, feeding their custom L3 database. (This isn't in the requirements right now -- we say that images will become available in 24hrs -- and is addressed in Section \ref{ssec:dmplans_user}).