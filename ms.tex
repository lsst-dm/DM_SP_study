\documentclass[DM,lsstdraft,toc]{lsstdoc}
\usepackage{graphicx}
\usepackage{url}
\usepackage{latexsym}
\usepackage{color}
% black, blue, brown, cyan, darkgray, gray, green, lightgray, lime, magenta, blue, orange, pink, purple, red, teal, violet, white, yellow.

\title[LSST Special Programs]{Data Management \\ and LSST Special Programs}

\author{M.~L.~Graham}

\setDocRef{DMTN-nnn}
\date{\today}
\setDocRevision{TBD}
\setDocStatus{draft}

\setDocAbstract{\textbf{WORKING DRAFT } but intended to be an overview of any and all potential requirements on data management (DM) from special programs (SP) such as the deep drilling fields (DDF) or mini-surveys (MS). We also summarize the plans for how the special programs data will be integrated into the wide-fast-deep (WFD) Main Survey (Levels 1 and 2). It is the intent that this document evolve to become an internal catalog of change requests to the DM Level 1 and 2 pipelines and products, and a community resource for future special program white papers and preparations for Level 3 pipelines. \textbf{Preliminary action items are listed in Section \ref{ssec:intro_action}.} 

\medskip MLG Summary: The very high level requirements for Level 3 and Special Programs are written down in the DMSR, but they're broad. The next document down-flow, the LDM-151, is focused on the Level 1/2 algorithms with no accountability to the DMSR requirements related to SP and Level 3 (and that's fine). LDM-151, Section 7, mentions the DAX and SUIT, and then also mentions that it will be detailed in a different document. At this stage it seems it is maybe a mite too early to conclude anything, but we can go through the "Action Items" and try to resolve the questions that exist, for now.
}


\setDocChangeRecord{%
\addtohist{1}{2017-04-??}{Internal working document.}{Melissa Graham}
%\addtohist{2}{yyyy-mm-dd}{Future changes}{Future person}
}

\begin{document}

\maketitle

% CITATION EXAMPLES
% \verb|\citellp|: \citellp{LPM-17, LSE-30} \\
% \verb|\citell|: (SRD; \citell{LPM-17,LSE-29}) \\
% \verb|\citep[][]|: \citep[e.g.,][are interesting]{LPM-17,LSE-29} \\
% \verb|\cite|: \cite{LPM-17,LSE-29}



% % % % % % % % % % % % % % % % % % % % % % % % % % % % % % % % % % % % 
\section{Introduction} \label{sec:intro}

\noindent The current status of this work is that it is an internal, ongoing study with several interim preliminary action items or outstanding questions so far (Section \ref{ssec:intro_action}).

\noindent The purpose of this study is to: \\
(1) summarize the current plans of the DM pipelines with respect to special programs, \\
(2) summarize the processing that is required to enable science for special programs, \\
(3) assess whether the DM pipelines meet the expected needs of special programs, and \\
(4) ensure any necessary changes get written into the requirements, designs, and plans.

\noindent This will be accomplished by: \\
(1) starting with the DPDD, whitepapers, etc., and compiling relevant information, \\
(2) discussing initial issues internally with DM (Mario, K-T, Tim), and then \\
(3) ensure the identified needs become specified requirements (i.e., action items Section \ref{ssec:intro_action}).

{\it Mario: "Examples include making sure there's a requirement that the system is able to use an arbitrary template for some (deep drilling) field and a plan to deliver that functionality; or that the requirements define how a user-supplied code will be able to run in the batch system to process the last-night's data with shift-and-stack code for KBOs."}

\subsection{Current Action Items}\label{ssec:intro_action}

Action items for Data Management can be done internally as needed -- some are just questions of things to investigate. Items that involve gather feedback from the Science Collaborations should probably wait and happen at the same time as the next calls for community input (see Section \ref{ssec:intro_forum}).

\noindent \textbf{Data Management: } \\
\textbf{(1)} Confirm whether processed single visits for Special Programs can stay on disk and accessible for Level 3 pipelines for a longer amount of time than the WFD images, by user request. (Section \ref{ssec:dmdocs_review}) \\
\textbf{(2)} Confirm if the budget requires additional computational resources for Special Programs and Level 3 processing. (Section \ref{ssec:dmdocs_review}) \\
\textbf{(3)} Find out the thermal tolerances and determine the minimum and maximum exposure times (and/or cadence of short exposures) that will satisfy it. (Section \ref{ssec:dmdocs_review}) \\
\textbf{(4)} Find out what the expected filter change time will be (maximum is 120s). (Section \ref{ssec:dmdocs_review}) \\
\textbf{(5)} Confirm whether alternative products can contribute to the Alert Stream. (Section \ref{ssec:issues_alerts}) \\
\textbf{(6)} Update the database schema for {\tt DIASource} to add an element that contains information about which template was used to create the difference image (e.g., {\tt tractId} or {\tt patchId}). (Section \ref{ssec:issues_databases}) \\
\textbf{(7)} Clarify whether database schema key {\tt Object.prv\_inputID} will identify whether an object is an externally provided coordinate for forced photometry. (Section \ref{ssec:issues_databases}) \\
\textbf{(8)} Find out whether a SP visit with a single exposure (no snaps) disqualifies it from incorporation into Level 1. (Section \ref{ssec:issues_snaps}) \\
\textbf{(9)} Confirm that the ISR algorithms will cover a range of exposure times and dither pattersn. (Section \ref{ssec:issues_calibrations}) \\
\textbf{(10)} Implanting fake sources remains TBD. (Section \ref{ssec:issues_calibrations}) \\
\textbf{(11)} Confirm that science users could write a Level 3 SAS with the existing planned algorithms? (Section \ref{ssec:issues_mops}) 

\noindent \textbf{Science Collaborations: } \\
\textbf{(1)} SC should be solicited to provide input about all special program science goals that require 60 second alert latency from non-standard exposure times. (Section \ref{ssec:issues_alerts})\\
\textbf{(2)} SC should be solicited to provide input on whether the space allotted for object characterization parameters in the database schema such as variability and morphology are adequate to describe objects with observations over a wider range of cadences and depths than the WFD observations. E.g., can the variability parameter characterize fast cadence data, does the morphology parameter appropriately identify low-surface brightness features. DM will first compile and provide a list of planned parameters from the schema. (Section \ref{ssec:issues_databases}) \\
\textbf{(3)} SC should be solicited to provide input on whether there are science use-cases for specialized, auto-generated template images e.g., with constant $\Delta t$, or image-image differencing (instead of image-template). DM should then assess whether this will require additional database elements. (Section \ref{ssec:issues_templates}) \\



\subsection{Interacting with the Science Community} \label{ssec:intro_forum}

Several of the Action Items suggest soliciting input from the science collaborations (SC). However, it may be wiser/easier to wait until the next call for community input, and package these requests up in a broader request that a kind of "DM Requirements" section be included in all white papers. The document "Cadence Exploration and Improvement Plan" (by Zeljko, Lynne, Andy) proposes that the next call for DDF White Papers (proposed fields and cadences) be in December 2017, and the call for Mini-Survey White Papers be in October 2018. 

\noindent Recall that there is already a community forum on this topic, so far minimally used but ready to be revised in the future when the next calls soliciting white papers are released. \\ \url{http://community.lsst.org/t/deep-drilling-fields-and-data-management/1115}.

\noindent Remember also that the SC Coordinator for DDF: Neil Brandt.

\noindent The questions opened up on that forum are: \\
(1) What additional processing beyond that currently planned by the DM team (alerting relative to an annually created template) would greatly enhance the DDF science goals? \\
(2) Are there DDF or Mini-Surveys specific aspects of the Level 3 system that would add significant value if provided? ``Level 3" is the LSST-provided capability that enables non-DM, user-driven, processing of LSST data at the LSST Archive center (or remotely). \\
(3) Are there aspects of the Science User Interface and Tools (SUIT) that need to be developed in order to enhance the usefulness of DDF data products. \\
(4) To what degree should the DDF or Mini-Survey imaging could/should be incorporated into the main survey's deep stacks and associated data products (as opposed to being processed as separate data products)?



% % % % % % % % % % % % % % % % % % % % % % % % % % % % % % % % % % % % 
\clearpage
\section{The Current DM Plans for Special Programs} \label{sec:dmdocs}

In this section we review what is written down in LSST documents with regards to the processing of Special Programs data. Below, we review the relevant contents of the three Data Management documents, and then the non-DM LSST documents. 

\textbf{Overview:} Although the data from some special programs might be incorporated into the Level 1, 2, and/or MOPS pipelines (see Section \ref{ssec:dmdocs_SPinWFD} below), every special program will also have one or more associated Level 3 processing pipelines. These Level 3 processing pipelines will be assembled by the science users, hopefully using existing LSST codes and schemas (so that e.g., the Level 3 databases that can be joined to the Level 1/2 databases), and also with custom codes. It is clearly specified that DM will not write any specialized algorithms for Special Programs data, but that an effort will be made to ensure the DM codes are extendable to Level 3. DM will also ensure that the relevant software infrastructure exists for science users, and for LSST to automatically run user-provided Level 3 pipelines (e.g., difference imaging on deep drilling nightly stacks). To enable science goals, some Level 3 pipelines can be run by LSST with a projected latency of (at most) tens of minutes (as opposed to the 60 second guaranteed for WFD). Level 3 pipelines for stacks and co-adds that do not have the same real-time urgency can simply be run by users through the Science Platform. 


\subsection{Documentation Review}\label{ssec:dmdocs_review}

\noindent \textbf{DMSR, LSE-61} \\
Data Management Subsystems Requirements \citep{LSE-61} contains the high-level requirements for all processing and products. These have flowed down from the SRD (LPM-17), the SR (LSE-29), and the OSS (LSE-30), which are discussed in the next section. \\
$\bullet$ The only part of the DMSR that directly references Special Programs (or DDF or mini-surveys) is Section 1.2.2 "Processed Visit Images" (ID: DMS-REQ-0069), which specifies that "Processed science exposures are not archived, and are retained for only a limited time to facilitate down-stream processing. They will be re-generated for users on-demand using the latest processing software and calibrations", but also says that "This aspect of the processing for Special Programs data is specific to each program". \\
$\rightarrow$ \textcolor{red}{It is unclear whether processed SP data will remain on disk and accessible when required by the science users, or if it will only live a short while}. \\
$\rightarrow$ \textbf{Confirm whether processed single visits for Special Programs can stay on disk and accessible for Level 3 pipelines for a longer amount of time, by user request.} \\
$\bullet$ Section 2.9, "Level 3 Production", specifies high-level services of the DMS for Level 3 processing, which are relevant to special programs. \\
2.9.1. Level 3 Data Import (ID: DMS-REQ-0290): to provide the ability to ingest common file formats (e.g., FITS table of external catalog). \\
2.9.2. Resource Allocation (ID: DMS-REQ-0119): to provide a mechanism to prioritize Level 3 activities and appropriately allocate processing resources. \\
2.9.3. Data Product Self-Consistency (ID: DMS-REQ-0120): to provide a means for ensuring Level 3 tasks can be carried out on self-consistent inputs. $\rightarrow$ \textcolor{red}{(?).} \\
2.9.4. Provenance (ID: DMS-REQ-0121): to provide a means for recording information about processing performed at DACs (DMS-provided input and user inputs) to help users work in a reproducible way. \\
2.9.5/6. Software Frameworks (ID: DMS-REQ-0125,DMS-REQ-0128): to provide a means for applying user-provided processing to catalog and image data, and for assessing the completeness of that application. \\
3.1. Software for Community Re-Use (ID: DMS-REQ-0308): algorithms shall be designed to run on both high-performance platforms and users' desktops. $\rightarrow$ \textcolor{red}{I thought that the Science Platform was being developed specifically so that no one needs to process LSST data on their home machines?} \\

\noindent \textbf{DMAD, LDM-151}\\
LSST Data Management Applications Design \citep{LDM-151} outlines the code and algorithms that will be used in the processing of images in the AP (Level 1), DRP (Level 2), and MOPS. As such, it does not say much about Special Programs. It does say that "LSST DM is required to facilitate the creation of Level 3 data products by providing suitable APIs, software components, and computing infrastructure, but will not by itself create any Level 3 data products." The latter meaning that DM will not create any processing algorithms that are specific to the needs of any Special Programs data. Section 7 briefly describes the Data Access and the Science User Interface and Tool (SUIT), also known as the Science Platform, with a mention that they will be described in full in later documents. The Data Access layer (DAX) will be for sharing, accessing, searching (etc.) products, and the SUIT/Platform will provide an interactive environment for users, and also serve as portal for Level 3 pipeline development (including installed version of the LSST Software Stack), execution, retrieval, and analysis.

\noindent \textbf{DPDD, LSE-163}\\
Data Products Definitions Document \citep{LSE-163} specifies that the data products for Special Programs will ``be created using the same software and hardware as Levels 1 and 2". Section 7 is devoted to the data products of Special Programs, explaining how the current plan is to have both the Level 1 and 2 processing pipelines run on SP data in the same way that they run on the WFD, and produce all of the same data products and alerts, when possible/desirable. The Level 3 products for SP data can/will be kept in separate (joinable) databases that have identical schema as the WFD databases. \\
$\bullet$ It is specified that SP data processing will be limited to not use more than 10\% of the DM computational and storage facilities. During the telecon on Wed Mar 29, KT Lam reviewed the computational budget but it seemed mainly for WFD with only passing mention of DDF or special programs. \\
$\rightarrow$ \textcolor{red}{10\% of the computational budget seems like an underestimate. Although SP data will not be more than 10\% of the total data volume, it will be multiply processed: once as WFD Level 1, again perhaps as Level 2, and then possibly in multiple pipelines as Level 3, some of which are extremely computationally intensive (e.g., shift-and-stack). On top of this, add the Level 3 processing that will be applied to WFD data, and we could easily find that the Special Programs data are processed $\sim4\times$ as much as WFD.} \\
$\rightarrow$ \textbf{Confirm if the budget requires additional computational resources for Special Programs and Level 3 processing.}


\noindent \textbf{SRD, LPM-17}\\
The Science Requirements Document \citep{LPM-17} makes no reference to special programs, deep drilling fields, or mini-surveys; it's scope is the science of the WFD only.

\noindent \textbf{LSST SR, LSE-29}\\
The System Requirements document \citep{LSE-29} is derived from the LSST Science Requirements Document (SRD, \citep{LPM-17}). It contains the high-level specifications about fraction of total observing time delegated for special programs, minimum exposure times, maximum filter change time, etc., that flow down to the DMSR and OSS (detailed and discussed above and below).

\noindent \textbf{OSS, LSE-30}\\
The Observatory System Specifications (OSS) document \citep{LSE-30} contains high-level requirements for the Level 3 data products, and their storage and access, that are similar to those in the DMSR (as they both stem from LPM-17 and LSE-29). There are two points worth mentioning: \\
$\bullet$ 3.6.1.4. Minimum exposure time (ID: OSS-REQ-0291): 1 second (stretch 0.1 seconds). \\
$\rightarrow$ \textcolor{red}{There is a note that if the exposure time is shortened from 15 seconds, the spacing between exposures may need to be lengthened in order to maintain camera thermal stability, and that thermal stability might also be affected with longer exposure times.}\\
$\rightarrow$ \textbf{Find out the thermal tolerances and determine the minimum and maximum exposure times (and/or cadence of short exposures) that will satisfy it.} \\
$\bullet$ 3.6.2.2. Maximum time for operational filter change (ID: OSS-REQ-0293): 120 seconds \\
$\rightarrow$ \textcolor{red}{This maximum time for operational filter change would be a debilitating overhead for a deep drilling cadence of consecutive 30 second images that rotates through filters.}\\
$\rightarrow$ \textbf{Find out what the expected filter change time will be.} \\
$\bullet$ 3.6.2.4. Minimum filter change count (ID: OSS-REQ-0295): day, 8, night, 4 \\
$\rightarrow$ \textcolor{red}{There is no related specification limiting the maximum number of total filter changes to avoid wear-and-tear on the mechanism?} 

\noindent \textbf{LSE-180}\\
Level 2 Photometric Calibration for the LSST Survey \cite{LSE-180} is built on OpSim runs that do include some nominal deep drilling fields, but the photometric calibration investigated in this work does not deal with potential issues induced by non-standard visit patterns or exposure times of special programs -- it's scope is that of the WFD Main Survey. 


% % % % % % % % % % % % % % % % % % 
\subsection{Incorporating SP Data into the WFD Main Survey}\label{ssec:dmdocs_SPinWFD}

I haven't found much written down in terms of guidelines or expectations for incorporating Special Programs data into the Level 1 and Alert Stream, Level 2 DRP, and/or MOPS pipelines -- especially when these data deviate from the standard 30 second visit -- but it seems feasible. Here we make these considerations.

\noindent \textbf{Incorporating SP data into Level 1 and the Alert Stream.} \\
It is reasonable to attempt to include as many LSST images into Level 1 and the Alert Stream as possible. When the exposure time is equivalent to a WFD visit image, $\sim 30$ seconds, this is obviously fine. When the exposure time is shorter than a WFD visit image, this should be fine as well. The minimum supported exposure time is currently specified as 1 second, which spans a dynamic range of $r \approx 12.9$ -- $21.0$ magnitudes. A template image built on $15$ second exposures will saturate at $r \approx 15.8$, but this still leaves stars between $15.8$--$21.0$ magnitudes to be used in the PSF-matching (all other filters have a similarly large overlap). The exposure times of a special program image might be longer than the WFD visit image, but given that the template will typically be a stack of a year or two of data (at least), and that it's hard to imagine a reason for even doing $150$ second exposures (given that the saturation would be $r\approx 18.3$ and the cosmic rays would intensify), it seems unlikely that any images would be rejected from Level 1 for having too long an exposure time. This means that the limiting factor of including any/all special programs data into the Level 1 and Alert Stream will be a processing back-up caused by the higher cadence of short exposures, or a lack of template images for special programs that go outside the WFD search area.

\noindent \textbf{Incorporating SP data into MOPS.} \\
Needs and capabilities are currently undefined.

\noindent \textbf{Incorporating SP data into the Level 2 DRP.} \\
The core science needs for uniform-depth images and catalogs for the WFD main survey probably means this will not happen, but identical Level 3 data pipelines and products may be used for regions with deeper coverage and the two databases can be joined.




% % % % % % % % % % % % % % % % % % % % % % % % % % % % % % % % % % % % 
\clearpage
\section{Questions and Potential Issues}\label{sec:issues}

In this section we list questions or potential problems identified only from reading LSST documents \citep{LSE-163,LDM-151}, or issues raised also in Gregory's AHM 2016 slides\footnote{``DM Considerations for Deep Drilling", LSST AHM Aug 2016 presentation by Gregory Dubois-Felsmann \url{https://zenodo.org/record/61402\#.WNVk6hIrIUF}}. Another useful resource is the LSST Database Schema Browser: \url{http://lsst-web.ncsa.illinois.edu/schema/index.php?sVer=baseline}.

For each of these questions or potential issues, a response will be: \\
\textcolor{blue}{blue if the answer is known} (even if the answer is a negative), \\
\textcolor{red}{red if the answer is still to be determined}, and \\
\textbf{bold if an action item is required}.

% % % % % % % % % % % % % % % % % % 
\subsection{Alert Stream}\label{ssec:issues_alerts}

$\bullet$ Will the near-real-time (tens of minutes) difference-image processing of Level 3 pipelines be set up to contribute to the same Alert Stream as Level 1? (E.g., transient source detections in nightly stacks of a deep drilling field.) \\
$\rightarrow$  \textcolor{red}{As it is described in LDM-151, the Alert Stream takes as input only {\tt DIAObjects} and specifies that {\it "all alerts should be transmitted within 60s of the closure of the shutter of the final snap within a visit."} Based on this, any Level 3 difference imaging sources would not contribute to the Alert Stream; only imaging that qualifies as Level 1 will contribute to the Alert Stream.} \\
$\rightarrow$ \textbf{Confirm whether alternative products can contribute to the Alert Stream.}

$\bullet$ When we attempt to include short exposures (e.g., 5 seconds) in Level 1 then this higher data acquisition rate (e.g., $\sim$6$\times$) might cause a processing back-up at NCSA. Is there a science need for 60 second latency on a consecutive time-series of short exposures? Or would e.g., 60 seconds latency on every 6$^{\rm th}$ image, be preferred? \\
$\rightarrow$ \textbf{Query the community about all special program science goals that require 60 second alert latency from non-standard exposure times.}

$\bullet$ Alerts on a {\tt DIASource} will link to the associated {\tt DIAObject}, but in cases where a special program obtains a sequence of images without switching fields, will the {\tt DIAObject} catalog have had enough time to be updated to include the {\tt DIASource} from the previous image? \\
$\rightarrow$  \textcolor{blue}{Immediate updates of the {\tt DIAObject} catalog is an expected capability, although it is currently uncertain how it will be accomplished, technically.} 

$\bullet$ LDM-151, Section 3.3.4 `Alert queuing and persistance' mentions that the {\it ``event message stream and the AlertDB will be synchronized at least once every 24 hours"} and that {\it ``Prior to the start of the subsequent night's observations, the message queue will be flushed and synchronized with the AlertDB. It is possible to persist the message queue on longer timescale but it is a requirement that synchronization be performed within 24 hours of the observations."} Since this is for the main survey in which new alerts are issued for a given field 2--3 times per night, check that this is fine in the case of doing e.g., 200 alert bursts on the same field in a single night from a deep drilling field. \\
$\rightarrow$  \textcolor{blue}{Upon review, this requirement seems unrelated to visit locations, and thus not an issue for special programs.} \\


% % % % % % % % % % % % % % % % % % 
\subsection{Databases and Schema}\label{ssec:issues_databases}

$\bullet$ The database schema for {\tt DIASource} needs to have an element added to identify the template image that was used; this is relevant for the Main Survey WFD Level 1 and also all Level 3 differencing products. \\
$\rightarrow$  \textbf{Database schema for {\tt DIASource} needs to have an element added that contains information about the template that was used to create the difference image.}\\

$\bullet$ LDM-151 mentions that forced photometry will be done for externally defined targets (Section 3.2.5), and this may be a particular interest to special programs, but it is unclear how such targets will be identified or flagged as such in the databases. \\
$\rightarrow$ \textcolor{red}{This is definitely a promised Level 1 product, but it is unclear whether we need to add an element to the database schema. Currently, the {\tt Object} database contains an element {\tt prv\_inputId} which is an {\tt integer}, and is described as the {\it ``Pointer to prv\_InputType. Indicates which input was used to produce a given object."} Is that all we need?} \\
$\rightarrow$  \textbf{Clarify whether {\tt Object.prv\_inputID} will identify whether an object is an externally provided coordinate for forced photometry.}

$\bullet$ Will the following element sizes be enough the characterize the different kinds of variability that could be measured with cadences that are quite different from the main survey (i.e., longer or shorter cadenced special programs)? \\
{\tt Object} and {\tt DIAObject.lcPeriodic} = {\tt float[6 x 32]} = Periodic features extracted from light-curves using generalized Lomb-Scargle periodogram \\
{\tt Object} and {\tt DIAObject.lcNonPeriodic} = {\tt float[6 x 32]} = Non-periodic features extracted from light-curves using generalized Lomb-Scargle periodogram \\
$\rightarrow$ \textcolor{red}{Although these elements could be differently populated in the Level 3 databases in order to meet the science goals of a given special program, we need to ensure that they are adequate for the Level 1 data which might incorporate many timescales.}\\
$\rightarrow$  \textbf{Contact the Science Collaborations for input on the characterization parameters for variability.}

$\bullet$ Low surface-brightness features and/or objects are required for some galaxy science, and will show up more in e.g., a DDF than the WFD stack. Will the related shape parameters such as extendedness that will be included in the {\tt Object} database be sufficient to describe the extra level of detail in low-surface brightness features afforded by e.g., DDF stacks? \\
$\rightarrow$ \textcolor{red}{As above, these elements could be differently populated in the Level 3 databases in order to meet the science goals of a given special program. It's unlikely that extremely deep patches will be included in the Level 2 products, but this needs assessment.}\\
$\rightarrow$  \textbf{Contact the Science Collaborations for input on the characterization parameters for extended objects.}

$\bullet$ Images obtained during special programs might be used with one or more different types of templates -- the same images could be used as multiple surveys. Is it feasible to have separate databases for each? \\ 
$\rightarrow$ \textcolor{blue}{Yes, Level 3 pipelines can create as many separate databases as necessary. There should at least be a separate database for each kind of template image. These databases can have the same base schema, and elements of the schema can be ``turned off" if the survey will not use them, to save processing time and disk space.} \\

$\bullet$ Data from special programs might be incorporated into the Level 1 and/or 2 databases; if so, how will it be flagged in these databases? \\
$\rightarrow$  \textcolor{blue}{In the schema for {\tt DIASource} and {\tt Source} there is an element for {\tt ccdVisitId}, and the database {\tt CcdVisit} has entry for {\tt visitId}, and the database {\tt Visit} has an entry for {\tt programId}, which is currently an {\tt integer} containing the {\it ``Observing program id (e.g., universal cadence, or one of the deep drilling programs, etc.)."}.} \\

$\bullet$ Will a Level 3 database for a Special Program be able to generate and incorporate forced photometry from WFD images? \\
$\rightarrow$ \textcolor{blue}{Yes. All Level 3 pipelines have access to all LSST data.}

$\bullet$ Will the Level 1 requirement to generate forced photometry for all new {\tt DIAObjects} within 24 hours be able to run on Level 3 image products of Special Programs data, such as deep nightly stacks of DDF images? \\
$\rightarrow$ \textcolor{blue}{This seems unlikely to be included as Level 1 (and unnecessary). Performing forced photometry for new {\tt DIAObjects} on a Level 3 image product would itself be a Level 3 pipeline.}


% % % % % % % % % % % % % % % % % % 
\subsection{Templates}\label{ssec:issues_templates}

$\bullet$ Typically we think of a template as a single deep coadd of image obtained e.g., $\Delta t \geq 1$ year ago. However, the use of a template with constant $\Delta t$ to the new images (e.g., $\Delta t = 7$ days to mitigate proper motion induced issues in a difference image for a Galactic plane mini-survey), or the use of a template that is simply a specific other image, might be scientifically necessary. Will this be supported by DM? \\
$\rightarrow$ \textcolor{blue}{Difference imaging with a custom template is Level 3, and the requirement for the generation of custom templates already exists as} \textbf{DMS-REQ-0032:} The DMS shall provide software to perform image differencing, generating Difference Exposures from the comparison of single exposures and/or coadded images \citep{LSE-61}.


% % % % % % % % % % % % % % % % % % 
\subsection{Snap Handling}\label{ssec:issues_snaps}

$\bullet$ LDM-151 says (Section 3.2.4.1) that the image differencing pipeline will measure the PSF flux on snap difference images for all {\tt DIASources}, but is there a lower limit to exposure time for this? E.g., would not apply to a 1s twilight survey? \\
$\rightarrow$ \textcolor{red}{Have not yet tried to answer this question, but I wonder if for Special Programs that have single-image visits instead of two snaps, whether the lack of snaps will disqualify them from Level 1 inclusion?}\\
$\rightarrow$ \textbf{Does a SP visit with a single exposure (no snaps) disqualify it from incorporation into Level 1?}


% % % % % % % % % % % % % % % % % % 
\subsection{Calibrations}\label{ssec:issues_calibrations}

$\bullet$ The Special Programs data may have a different exposure time or dither pattern than the WFD survey. The concern here is that if the instrument signature removal (ISR) calibration pipeline, and/or the photometric/astrometric calibration pipeline, have been designed to work for WFD data that problems might be encountered for Special Programs data (e.g., dark frames; brighter-fatter correction; artifacts with an exposure time dependence; charge build-up from no-dither with bright stars). \\
$\rightarrow$ \textcolor{red}{Since there already exists a requirement that DM deliver processed visit images (ID: DMS-REQ-0069) to users (and to internal pipelines), it seems that at least the ISR will be characterized for a variety of exposure times -- but will it for sure?}. \\
$\rightarrow$ Note that LSE-180 mentions that it is assumed that all factors affecting the system transmission are stable on 15 second timescales (page 10), but not what the upper limit of that might be. \\
$\rightarrow$ LSE-180 comments on the dither pattern for the WFD survey in that "dither patterns where the overlap is one quarter of the field of view or more produce results meeting the SRD requirements", but this is specific to photometric calibration of the WFD. The LSE-180 also mentions that an inappropriate dither pattern can make it hard to correct for the variation of system bandpass as a function of the focal plane position -- but so long as this is solved in the WFD the corrections can be applied to the much smaller amount of data from the special programs (unless it is a function of exposure time and the special program uses a different exposure time). \\
$\rightarrow$ \textbf{Confirm whether it is implicit in the LDM-151 that all the ISR algorithms will automatically investigate and derive corrections for a range of exposure times, dither patterns, etc.}

$\bullet$ Planting fake sources may be both more manageable and more important to some special programs. How will DM support this? (Answer needs to distinguish between planting fakes in the single images for detection efficiencies vs. planting fakes in the coadds for point-source limiting magnitudes). \\
$\rightarrow$ The word 'fake' appears just once across LDM-151, LSE-61, and LSE-180 combined, and it is in Section 5.6.3 'MakeSelectionMaps' of LDM-151, which comments on how the depth and detection efficiency maps will be created, suggesting that fake implanting is one way. \\
$\rightarrow$ \textcolor{red}{Everything related to fake sources remains very unclear, even at Level 1 and 2.} \\
$\rightarrow$ \textbf{Figure out fake implantation?}

$\bullet$ Multiple special programs request short-exposure survey, which would provide less/different stars for photometric/astrometric calibration. Would a 1s exposure contain enough stars for calibration? \\
$\rightarrow$ \textcolor{blue}{A 1 second exposure will contain stars from $12.9<r<21.0$, which should be more than enough for a full photometric calibration.} \\
$\rightarrow$ Note that LSE-180 only considers calibration for the standard WFD data.

% % % % % % % % % % % % % % % % % % 
\subsection{Custom Co-Adds}\label{ssec:issues_coadds}

$\bullet$ Building deep co-adds with no transient contamination (e.g., SN hosts, cosmology uses host parameters correlated with intrinsic SN brightness), or creating deep stacks across filters for faint-object detection at $<5\sigma$ in a single filter. \\
$\rightarrow$ \textcolor{blue}{Creating custom co-adds will be a Level 3 capability that will be possible with the Level 2 DRP pipeline algorithms described in LDM-151. This includes multi-filter stacks (DMSR Section 1.3.4. 'Multi-band Coadds' ID:DMS-REQ-0281).}


% % % % % % % % % % % % % % % % % % 
\subsection{Moving Objects}\label{ssec:issues_mops}

$\bullet$ For faint moving objects, users will develop shift-and-stack (SAS) pipelines in Level 3 (MOPS does not SAS). Will they have all the necessary infrastructure for this intensive process? Will they be able to use the LSST real/bogus on their shift-and-stacked frames? \\
$\rightarrow$ \textcolor{red}{In particular, people may want to SAS on difference images, which would require DM making these available for some time.} \\
$\rightarrow$ \textbf{Confirm that science users could write a Level 3 SAS with the existing planned algorithms?}

$\bullet$ Can MOPS link newly discovered {\tt sources} to those discovered earlier on the same night (e.g., in a 4-visits-per-night scenario), or does it have a similar potential problem to {\tt DIASource} and {\tt DIAObject}? \\
$\rightarrow$  \textcolor{blue}{MOPS does not run during the night in real time. This is not a problem.}.


\clearpage
% % % % % % % % % % % % % % % % % % % % % % % % % % % % % % % % % % % % 
\section{Science Cases and their DM Requirements} \label{sec:science}

In this section we review science documents from LSST and the community. We compile information about the nominal observing plans for special programs that are already written down (Section \ref{ssec:science_plans}), and summarize the science motivations for special programs that have so far been proposed (Section \ref{ssec:science_descriptions}). Potential issues for DM are inferred from these science motivations and have been incorporated into our list of issues in Section \ref{sec:issues}.

\noindent Resources: \\
$\bullet$ ``LSST: from Science Drivers to Reference Design and Anticipated Data Products" \cite{2008arXiv0805.2366I} \\
$\bullet$ The LSST Science Requirements Document \cite{LPM-17} \\
$\bullet$ ``General Review of the Proposed DDF and MS", LSST AHM Aug 2016 presentation by Niel Brandt \url{https://project.lsst.org/meetings/lsst2016/sites/lsst.org.meetings.lsst2016/files/Brandt-DDF-MiniSurveys-01.pdf} \\
$\bullet$ ``Simulations, Metrics and Merit Function for Mini-Surveys and DDF", LSST AHM Aug 2016 presentation by Stephen Ridgway \url{https://project.lsst.org/meetings/lsst2016/sites/lsst.org.meetings.lsst2016/files/Ridgway-SimulationsMetrics_1.pdf}\\
$\bullet$ ``LSST's DC [Deep CoAdd] Bias Against Planets and Galactic-Plane Science" by A. Gould, \cite{2013arXiv1304.3455G} \url{https://arxiv.org/abs/1304.3455} \\
$\bullet$ Chapter 10 ``Special Surveys" of the Observing Strategy White Paper \cite{OSWP} \\
$\bullet$ List of LSST Deep Drilling white papers: \url{https://project.lsst.org/content/whitepapers32012} \\


\subsection{Nominal Observing Plans for DDF and MS}\label{ssec:science_plans}

\cite{2008arXiv0805.2366I}, Section 3.2.1 ``Mini-Surveys": describes a nominal DDF data set as $\sim50$ consecutive $15$ second exposures in each of four filters in one hour per night, once every two nights, for four months. Each observation would have a limit of $r\sim24.5$; a one-hour nightly stack would have a limit of $r\sim26.5$; and and assuming a $60\%$ completion rate (weather), the four-month $\sim40$ hours stacked together with the $\sim180$ main survey visits would yield a limit of $r\sim28$. The LSST Science Requirements Document \cite{LPM-17} doesn't contain the terms ``deep drilling field" or ``mini-survey"; I assume that the SRD is only for WFD survey.

Four extragalactic deep drilling fields have already been specified (Table \ref{tab:ddfms}). From a scouring of mainly the presentations of Brandt and Ridgway at the 2016 AHM, an {\it incomplete} list of potential mini-surveys that people are thinking about in Table \ref{tab:ddfms} also.

\begin{table}[h]
\begin{center}
\begin{footnotesize}
\caption{Approved DDF and Incomplete List of Potential MS.}
\label{tab:ddfms}
\begin{tabular}{lll}
\hline \hline
Name & Coordinates & Description  \\
\hline
DDF Elias S1    & 00:37:48, -44:00:00  & approved, cadence TBD \\
DDF XMM-LSS & 02:22:50, -04:45:00  & approved, cadence TBD  \\
DDF Extended Chandra Deep Field-South & 03:32:30, -28:06:00  & approved, cadence TBD  \\
DDF COSMOS  & 10:00:24, +02:10:55 & approved, cadence TBD  \\
DDF TBD  & & TBD \\
North Ecliptic Spur      & & solar system objects (find and characterize) \\
Galactic Plane             & & more intensive stellar surveying \\
South Equatorial Cap  & & S/LMC and more Galactic science \\
Twilight                        & & short exposures (0.1s) for bright stars \\
Mini-Moons                     &  & finding mini-moons \\
Sweetspot                       & & 60 deg from Sun for NEOs on Earth-like orbits \\
Meter-Sized Impactors     & & detection a week before impact \\
GW Optical Counterparts & & search and recovery \\
Old Open Cluster M67      & dec +12 & compact survey above Galactic plane  \\
\hline
\end{tabular}
\end{footnotesize}
\end{center}
\end{table}


\subsection{Science Descriptions (Not Comprehensive)}\label{ssec:science_descriptions}

The following is a collection of rough notes on the science goals and potential data processing/product issues for some proposed special programs, classified by field. It is not comprehensive. The questions for DM that were inspired by considering the proposed special programs are still mingled in amongst these notes, but have also been incorporated into Section \ref{sec:issues}, and so the answers and solutions to any questions or problems are all in Section \ref{sec:issues} also.

\medskip
\noindent \textbf{Solar System Objects}\\
$\bullet$ Strategy would be to observe 3x3 grid of 9 fields centered on coordinate of an ecliptic latitude equal to 0 and a longitude at a conjunction of Trojan clouds of Jupiter and Neptune, 4 groups of 2 visits on a 2--3 day revisit schedule, with the 4 groups targeted at e.g., opposition, all done in $r$-band ($g$ suitable too). (Is this within the North Ecliptic Spur?) \citep{BeckerWP}. \\
$\bullet$ North Ecliptic Spur region will yield more $\geq140$ m NEOs (Brandt talk).  \\
$\bullet$ Are the three SSO special programs described in Chapter 10.2 of \cite{OSWP} mandated by NASA? \\
$\bullet$ SSO science goals require enabling the discovery of faint SSO via shift-and-stack of images in specific fields where SSO discovery is maximized \citep{BeckerWP}. \\
$\bullet$ Does MOPS processing do shift-and-stack, and/or is the MOPS database schema ready to incorporate a shift-and-stack output? \\
$\bullet$ Regarding shift-and-stack, \cite{BeckerWP} says that ``the multi-fit algorithm ... naturally provides a base infrastructure for this process. In particular, the marshaling of the pixels to attempt a given photometric measurement is non-trivial when tens of thousands of images are required. However, the multi-fit middleware is required to do exactly this, so we expect that this issue will be resolved by the time SAS is needed." True? \\
$\bullet$ To detect Earth impactors {\it ``hours before impact"} using the four visits per night will require real-time linking of new {\tt DIASource}s -- or maybe the MOPS handles this differently? \\

\noindent \textbf{Stars in the Milky Way and Magellanics} \\
$\bullet$ Special programs will provide images deeper than the main survey in order to identify useful stellar populations; getting the images over a short time span mitigates proper motion losses and increase flare detection rates \citep{DhitalWP}. \\
$\bullet$ Target Galactic plane regions and/or open clusters, typically 1 to 3 filters needed for detections of e.g., faint stars already detected in $z$ and $y$ but needing $i$ to distinguish from red galaxies \cite{DhitalWP}. \\
$\bullet$ Characterize stellar variability over the full L/SMC galaxies to $M_V<6.5$ on timescales from 15s to 3d \cite{SzkodyWP}. \\
$\bullet$ Special co-adds may be required, e.g., ``To reach variability levels of 0.1 to 0.005 mag will require co-adds depending on the timescale of the particular variables" \citep{SzkodyWP}. \\
$\bullet$ Light curve characterization parameters in the databases should include one for very short events like flares; ensure there's adequate space in the database for multiple variability parameterizations. \\
$\bullet$ How might LC-characterization parameters fail on e.g., an LBV-turned-SN? \\
$\bullet$ A Twilight Short Exposure survey would need good on-the-fly sky subtraction of a rapidly changing background; ensure plans are adeqaute? \\
$\bullet$ A short-exposure survey of M67, Chapter 10.4 of \cite{OSWP}, suggests using {\it ``custom pixel masks to accurately perform photometry on stars as much as 6 magnitudes brighter than the saturation level"}; feasible? \\

\noindent \textbf{Exoplanets} \\ 
$\bullet$ Transits. The nominal DDF plan described in \cite{2008arXiv0805.2366I} would allow for $1\%$ variability detection over hour-long timescales, which is suitable for detecting transits. A DDF field at Galactic latitude $30$ degrees would yield $10^6$ stars at $r<21$ that would have ${\rm SNR}>100$ in each single exposure of the sequence. Microlensing events can also be detected with this data set. In both cases, follow-up is required. The Galactic Plane MS is proposed for this. \cite{2013arXiv1304.3455G} describes how transits can be extract from the same data set.\\
$\bullet$ Microlensing. Slower than a transit, \cite{2013arXiv1304.3455G} suggests that $\sim22$ mag imaging every 3-4 days (i.e., the WFD nominal cadence) can find microlensing candidates (for follow-up with e.g., LCO). However, more Galactic regions must be included, and this will require image differencing in crowded fields. \cite{2013arXiv1304.3455G} claims that imaging in the necessary regions of the galaxy has been disfavored by the project on the basis that the eventual deep co-adds would be uselessly confusion limited. I don't think DM could fix this fact, and deep co-adds (and all DRP) could possibly just be skipped for these regions. Otherwise, the already-planned AP should work fine for Galactic plane survey.\\

\noindent \textbf{Supernovae:} \\
$\bullet$ The nominal DDF plan described in \cite{2008arXiv0805.2366I} would allow for nightly stacks with a limit of $r\sim26.5$, extending the SN sample to $z\sim1.2$ and providing more densely sampled light curves for cosmological analyses. \\
$\bullet$ \cite{CrottsWP} doesn't say much specific. \\
$\bullet$ \cite{KesslerWP} describes their optimal exposure time distribution as 6, 5, 10, 10, 9, 10 in $ugrizy$, which adds up to $<60$ minutes and could presumably be done with the standard 30s exposures. \\
$\bullet$ Template generation: e.g., supernovae will want a DDF template from $>30$ days ago (minimum for SNeIa); $>100$ days is better (SNII), and $>200$ days safest (SLSNe, SNIIn). So, last DRP will be fine. But some last longer, thousands of days, and so users will want to know the date of the template. \\
$\bullet$ Building special, deep-as-possible, SN-free host galaxy images in order to measure correlated properties and make corrections to the LC, and do host science. The regular DRP will be inadequate for this (this could be a Level 3 tool to stack requested epochs only?). With short runs on a DDF this might be impossible, would have to revisit.  \cite{FergusonWP} also mentions {\it ``characterization of ultra-faint SN host galaxies"} in their Galaxies WP. \\
$\bullet$ A short-exposure survey could include nearby SNeIa on the same photometric system -- but can we calibrate as well with the limited set of stars? \\

\noindent \textbf{Galaxies:} \\
$\bullet$ \cite{FergusonWP} mentions {\it ``identification of nearby isolated low-redshift dwarf galaxies via surface-brightness fluctuations"} and {\it ``characterization of low-surface-brightness extended features around both nearby and distant galaxies"} -- building a large collection of low-$\mu$ objects is a science driver but do these even get in the {\tt Object} database (\textbf{check on the `extendedness'-related parameters, e.g. {\sc Galfit} modeling? to what $\mu$?}), or is this assumed entirely Level 3? \\
$\bullet$ It seems that e.g., characterization of high-$z$ clusters will depend on ability to deblend extended objects. \\
$\bullet$ AGN monitoring in well-characterized galaxies \citep{FergusonWP}, and for a long-term DDF on many timescales \citep{GawiserWP} \\
$\bullet$ Effect of AGN on e.g., bulge-disk decomposition parameters? \\

\noindent \textbf{Weak lensing:} \\
$\bullet$ DDF can help with shear systematics and the effects of magnification in the analysis of WFD data (community forum, Jim Bosch) \\
$\bullet$ ``Will need to process at least some deep drilling fields (high-latitude ones) in the same way we process a full data release production before running the full data release production, so we can use the results to build priors and/or calibrate shear estimates on the wide survey" (community forum, Jim Bosch) \\
$\bullet$ Will need to process various wide-depth subsets of some deep drilling fields (again, high-latitude ones) using the regular DRP pipeline. We'll definitely want best-seeing, worst-seeing, and probably a couple of independent typical-seeing subsets, but there may be other ways we'd want to subdivide as well. (community forum, Jim Bosch)  \\
$\bullet$ photo-$z$ are very important \citep{MaWP} and so perhaps the implemented method should be chosen with weak lensing science prioritized \\





% % % % % % % % % % % % % % % % % % % % % % % % % % % % % % % % % % % % 
\section{Conclusions}\label{sec:conc}

Conclusions section.

%\acknowledgments
%Acknowledgments.
%
%\appendix
%Appendix.

\bibliography{ms,lsst,refs,books,refs_ads}

\end{document}



%\begin{center}
%\includegraphics[width=8cm]{figures/}
%\includegraphics[width=8cm]{figures/}
%\caption{ \label{fig:}}
%\end{center}
%\end{figure}










